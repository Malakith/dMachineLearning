
\documentclass[a4paper,10pt,article,oneside,english]{memoir} 
% DANSK OPSÆTNING
\usepackage[english]{babel}
\usepackage[utf8]{inputenc}
\usepackage[T1]{fontenc}
\usepackage{lmodern} 
\renewcommand{\englishhyphenmins}{22} 


% FIGURER
\usepackage{graphicx}

% MATEMATIK
\usepackage{amsmath}
\usepackage{amssymb,amsthm,bm}
\usepackage{mathtools}	


% This document uses
\usepackage[draft,silent]{fixme}
\usepackage{hyperref}
\usepackage{siunitx}
\usepackage{booktabs}

% captions in italic
\let\oldcaption\caption
\renewcommand{\caption}[1]{\oldcaption{\emph{#1}}}


\begin{document}
	%\frontmatter
	%\clearpage	
	%\tableofcontents*
	\title{Machine Learning E16 - Handin 4\\
		Representative-based Clustering Algorithms}
	\author{Group 22\\
		Mark Medum Bundgaard, Morten Jensen, Martin Sand Nielsen}
	\date{December 19, 2016, Aarhus University}
	
	\mainmatter
	\maketitle


\subsection{Requirements from Ira's description}\\
\begin{itemize}
	\item The status of the work, i.e., does it work, if not, then why.
	\item A discussion of plots of at least two runs of your algorithm implementations detailing what you can see. Make sure that you relate this to the discussion in the lecture or textbook about the strengths and weaknesses of the algorithms.
	\item A discussion of plots of the evaluation measures F1 and silhouette coefficient, detailing what you can learn from them. Include an explanation of what the evaluation measures reflect.
	\item Describe how you can use one of the clustering algorithms for image compression, and demonstrate the results for at least one algorithm on both images, discussing their quality and giving a reasoning for the differences.
\end{itemize}

\section*{K-keans clustering}

\section*{Gaussian Mixture-model Expectation Maximization}

\section*{Image color compression example}

\end{document}